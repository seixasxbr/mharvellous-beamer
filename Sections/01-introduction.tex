%*----------- SLIDE -------------------------------------------------------------
\begin{frame}[t]{Linha de Pesquisa}
    \framesubtitle{Robôs Aéros} 
    \transdissolve[duration=0.5]
    \large{
    \begin{itemize}
        \item Missão: Estudo das abordagens sobre robôs áereos autônomos, tendo como foco principal o desenvolvimento de técnicas de pouso e decolagens além de uma abordagem profunda nos controles de vôo.
        \vspace*{1cm}
        \item Justificativa: A popularização do uso de veículos aéreos em áreas como agricultura, cinematografia, militarismo e entretenimento.
    \end{itemize}
    % \begin{columns}
    %     \column{.02\textwidth}
    %     \column{.48\textwidth}
    %     \includegraphics[trim = 0 0 0 0, clip, width=1\textwidth,height=0.45\textwidth]{img1}

        
    %     \column{.49\textwidth}
    %     \includegraphics[trim = 0 0 0 0, clip, width=1\textwidth,height=0.45\textwidth]{img4}
    %     \column{.01\textwidth}
    %     \column{.01\textwidth}
        
    % \end{columns}
    }
%*----------- notes
    \note[item]{Notes can help you to remember important information. Turn on the notes option.}
\end{frame}
%-
%*----------- SLIDE -------------------------------------------------------------
% \begin{frame}[c]{} 
%     \transdissolve[duration=0.5]
   
%     \begin{columns}
%         \column{.01\textwidth}
%         \column{.49\textwidth}
%         \includegraphics[trim = 0 0 0 0, clip, width=1\textwidth,height=0.45\textwidth]{img1}
%         \!\\
%         \includegraphics[trim = 0 0 0 0, clip, width=1\textwidth,height=0.45\textwidth]{img2}

%         \column{.49\textwidth}
%         \includegraphics[trim = 0 0 0 0, clip, width=1\textwidth,height=0.45\textwidth]{img3}
%         \!\\
%         \includegraphics[trim = 0 0 0 0, clip, width=1\textwidth,height=0.45\textwidth]{img4}
%         \column{.01\textwidth}
%     \end{columns}
    
   
% %*----------- notes
%     \note[item]{Notes can help you to remember important information. Turn on the notes option.}
% \end{frame}
%*----------- SLIDE -------------------------------------------------------------
\begin{frame}[c]{Aplicações} 
    \transdissolve[duration=0.5]
   
    \begin{columns}
        \column{.01\textwidth}
        \column{.49\textwidth}
        \includemedia[
      width=1\linewidth,
      totalheight=0.8\linewidth,
      activate=pageopen,
      passcontext, 
      addresource=./Source/movies/agricultura1.mp4,
      flashvars={
      source=./Source/movies/agricultura1.mp4
      &autoPlay=true
      &autoRewind=true
      &Loop=true}
      ]{\fbox{\includegraphics{black}}}{VPlayer.swf}
        \column{.01\textwidth}

        \column{.49\textwidth}
        \includemedia[
      width=1\linewidth,
      totalheight=0.8\linewidth,
      activate=pageopen,
      passcontext, 
      addresource=./Source/movies/delivery1.mp4,
      flashvars={
      source=./Source/movies/delivery1.mp4
      &autoPlay=true
      &autoRewind=true
      &Loop=true}
      ]{\fbox{\includegraphics{black}}}{VPlayer.swf}
        \column{.01\textwidth}
        
    \end{columns}
    
   
%*----------- notes
    \note[item]{Notes can help you to remember important information. Turn on the notes option.}
\end{frame}
%*----------- SLIDE -------------------------------------------------------------
% \begin{frame}[t]{Projeto Carcará}
%     \transdissolve[duration=0.5]
%     \includegraphics[trim = 0 0 0 0, clip, width=1\textwidth,height=0.45\textwidth]{img4}
    
% %*----------- notes
%     \note[item]{Notes can help you to remember important information. Turn on the notes option.}
% \end{frame}
%-
%*----------- SLIDE -------------------------------------------------------------
\begin{frame}[t]{Projeto Carcará}
    \framesubtitle{A Plataforma}
    \transdissolve[duration=0.5]
    \begin{columns}
        \column{.01\textwidth}
        \column{.49\textwidth}
        \vspace*{0.4cm}
        \includegraphics[trim = 0 0 0 0, clip, width=1\textwidth]{CARCARAf.png}
        \column{.5\textwidth}
        \begin{itemize}
            \item Quadrotor - Cross configuration (X)
            \item Câmera frontal e inferior
            \item Sensores Ultrassônicos
            \item Laser
            \item Nvidia Jetson Nano
            \item Teensy 4.0
            \item IMU
        \end{itemize}
    \end{columns}
    
    
%*----------- notes
    \note[item]{Notes can help you to remember important information. Turn on the notes option.}
\end{frame}
%*----------- SLIDE -------------------------------------------------------------
\begin{frame}[t]{Projeto Carcará}
    \framesubtitle{Requisitos}
    \transdissolve[duration=0.5]
    \begin{columns}
        \column{.01\textwidth}
        \column{.49\textwidth}
        \begin{itemize}
            \item Framework: ROS 2
            \item SLAM
            \item Desvio de obstáculos
            \item Processamento de imagem em tempo real 
            \item Navegação autônoma
        \end{itemize}
        \column{.5\textwidth}
        \vspace*{0.4cm}
        \includegraphics[trim = 0 0 0 0, clip, width=1\textwidth]{CARCARAf.png}
    
    \end{columns}
    
    
%*----------- notes
    \note[item]{Notes can help you to remember important information. Turn on the notes option.}
\end{frame}
%-
%*----------- SLIDE -------------------------------------------------------------
\begin{frame}[t]{Projeto Carcará}
    \transdissolve[duration=0.5]
    \framesubtitle{Objetivo}
    \includegraphics[trim = 0 0 0 0, clip, width=1\textwidth,height=0.7\textheight]{objetivo4.png}
    
%*----------- notes
    \note[item]{Notes can help you to remember important information. Turn on the notes option.}
\end{frame}
%*----------- SLIDE -------------------------------------------------------------
\begin{frame}[t]{Projeto Carcará}
    \transdissolve[duration=0.5]
    \includemedia[
      width=1\linewidth,
      totalheight=0.45\linewidth,
      activate=pageopen,
      passcontext, 
      addresource=./Source/movies/carcara.mp4,
      flashvars={
      source=./Source/movies/carcara.mp4
      &autoPlay=true
      &autoRewind=true
      &Loop=true}
      ]{\fbox{\includegraphics{black}}}{VPlayer.swf}
    
%*----------- notes
    \note[item]{Notes can help you to remember important information. Turn on the notes option.}
\end{frame}
%*----------- SLIDE -------------------------------------------------------------
\begin{frame}[t]{Desafio da Linha de Pesquisa}
    \transdissolve[duration=0.5]
    \begin{columns}
        \column{.01\textwidth}
        \column{.49\textwidth}
        \vspace*{0.5cm}
        \includegraphics[trim = 200 0 300 100, clip, width=1\textwidth]{ardrone.png}
        \column{.49\textwidth}
        \Large{
        \begin{itemize}
            \item Plataforma: AR Drone 2.0
            \item Ambiente: Simulação
            \item Objetivo: Realizar seguimento de trajetória circular
            \item Framework: ROS Noetic
        \end{itemize}
        }
        \column{.01\textwidth}
    \end{columns}

    
%*----------- notes
    \note[item]{Notes can help you to remember important information. Turn on the notes option.}
\end{frame}
%-
% {
% \setbeamertemplate{background}
% {\includegraphics[trim = 0 0 0 0, clip, width = \the\paperwidth, height = \the\paperheight]{marcel-knupfer-37wuETxVwTQ-unsplash.jpg}}
% %*----------- SLIDE -------------------------------------------------------------
% \begin{frame}[c]{}
%     \transboxout[duration=0.5]
    
%     \begin{columns}
%         \column{.1\textwidth}
%         \column{.3\textwidth}
%         \column{.53\textwidth}
%             \vspace*{-3.5cm}
%             \Huge{\textbf{\textcolor{mracula5}{Quando chovia...}}}
%     \end{columns}
% %*----------- notes
%     \note[item]{Notes can help you to remember important information. Turn on the notes option.}
% \end{frame}
% %-
% }
% %*----------- SLIDE -------------------------------------------------------------
% \begin{frame}[t]{O sistema robótico}
%     \transboxout[duration=0.5]
%     \framesubtitle{Darwin-OP}
%     \begin{columns}
%         \column{.1\textwidth}
%         \column{.4\textwidth}
%             \includegraphics[width=.7\textwidth]{darwin-op}
%         \column{.4\textwidth}
%             \begin{enumerate}
%                 \item plataforma antropormórfica Darwin-OP;
%                 \item 20 DoF\footnote{do inglês, graus de liberdade};
%                 \item composto de 18 servo-motores;
%                 \item possui um grande gama de sensores para interação.
%             \end{enumerate}
%     \end{columns}
%  %*----------- notes
%     \note[item]{Notes can help you to remember important information. Turn on the notes option.}
% \end{frame}
% %-
% %*----------- SLIDE -------------------------------------------------------------
% \begin{frame}[c]{Darwin-OP - overview}
%     %\transboxin[duration=1,direction=30]
%     \centering

%     \includemedia[
%       width=0.7\linewidth,
%       totalheight=0.39375\linewidth,
%       activate=pageopen,
%       passcontext, 
%       addresource=./Source/movies/Darwin-OP.mp4,
%       flashvars={
%       source=./Source/movies/Darwin-OP.mp4
%       &autoPlay=true
%       &Loop=false}
%       ]{\fbox{\includegraphics{darwin-op}}}{VPlayer.swf}

% %*----------- notes
%     \note[item]{Notes can help you to remember important information. Turn on the notes option.}
% \end{frame}
% %-
% %*----------- SLIDE -------------------------------------------------------------
% \begin{frame}[t]{O sistema robótico}
%     \transboxout[duration=0.5]
%     \framesubtitle{Darwin-OP}
%     \begin{columns}
%         \column{.1\textwidth}
%         \column{.4\textwidth}
%         \column{.4\textwidth}
%     \end{columns}

%     \begin{block}{Um bloco de destaque}
%         Um exemplo de block.\\
%         Oferece um certo destaque.
%     \end{block}

%     \begin{alertblock}{Um bloco de destaque}
%         Um exemplo de alertblock.\\
%         Oferece um certo destaque.
%     \end{alertblock}

%     \begin{exampleblock}{Um bloco de destaque}
%         Um exemplo de exampleblock.
%     \end{exampleblock}
%  %*----------- notes
%     \note[item]{Notes can help you to remember important information. Turn on the notes option.}
% \end{frame}
% %-
% %*----------- SLIDE -------------------------------------------------------------
% \begin{frame}[t]{O sistema robótico}
%     \transboxout[duration=0.5]
%     \framesubtitle{PlantUML}
    
%     \tikzstyle{every node}=[draw=black,thick,anchor=west]
%     \tikzstyle{selected}=[draw=red,fill=red!30]
%     \tikzstyle{optional}=[dashed,fill=gray!50]

%     \begin{tikzpicture}[%
%         grow via three points={one child at (0.5,-0.7) and
%         two children at (0.5,-0.7) and (0.5,-1.4)},
%         edge from parent path={(\tikzparentnode.south) |- (\tikzchildnode.west)}]
%         \node {texmf}
%           child { node {doc}}		
%           child { node {fonts}}
%           child { node {source}}
%           child { node [selected] {tex}
%             child { node {generic}}
%             child { node [optional] {latex}}
%             child { node {plain}}
%           }
%           child [missing] {}				
%           child [missing] {}				
%           child [missing] {}				
%           child { node {texdoc}};
%       \end{tikzpicture}

%  %*----------- notes
%     \note[item]{Notes can help you to remember important information. Turn on the notes option.}
% \end{frame}
% %-
% %*----------- SLIDE -------------------------------------------------------------
% \begin{frame}[t]{O sistema robótico}
%     \transboxout[duration=0.5]
%     \framesubtitle{PlantUML}
    
%     % Define block styles
%     \tikzstyle{decision} = [diamond, draw, fill=blue!20, 
%     text width=4.5em, text badly centered, node distance=3cm, inner sep=0pt]
%     \tikzstyle{block} = [rectangle, draw, fill=blue!20, 
%     text width=5em, text centered, rounded corners, minimum height=4em]
%     \tikzstyle{line} = [draw, -latex']
%     \tikzstyle{cloud} = [draw, ellipse,fill=red!20, node distance=3cm,
%     minimum height=2em]

%     \begin{tikzpicture}[node distance = 2cm, auto]
%         % Place nodes
%         \node [block] (init) {initialize model};
%         \node [cloud, left of=init] (expert) {expert};
%         \node [cloud, right of=init] (system) {system};
%         \node [block, below of=init] (identify) {identify candidate models};
%         \node [block, below of=identify] (evaluate) {evaluate candidate models};
%         \node [block, left of=evaluate, node distance=3cm] (update) {update model};
%         \node [decision, below of=evaluate] (decide) {is best candidate better?};
%         \node [block, below of=decide, node distance=3cm] (stop) {stop};
%         % Draw edges
%         \path [line] (init) -- (identify);
%         \path [line] (identify) -- (evaluate);
%         \path [line] (evaluate) -- (decide);
%         \path [line] (decide) -| node [near start] {yes} (update);
%         \path [line] (update) |- (identify);
%         \path [line] (decide) -- node {no}(stop);
%         \path [line,dashed] (expert) -- (init);
%         \path [line,dashed] (system) -- (init);
%         \path [line,dashed] (system) |- (evaluate);
%     \end{tikzpicture}

%  %*----------- notes
%     \note[item]{Notes can help you to remember important information. Turn on the notes option.}
% \end{frame}
% %-
% %*----------- SLIDE -------------------------------------------------------------
% \begin{frame}[t]{O sistema robótico}
%     \transboxout[duration=0.5]
%     \framesubtitle{PlantUML}
    
%     \begin{tikzpicture}[line width=0.1pt]
%         \draw(0,0) circle(5cm);
%         \draw(0,0) circle(1cm);
%         \draw(0,0) node {\Huge$\mathbf{A}$};
%         \draw(0,0) circle(4.5cm);
%         \draw(-48:2.5) arc(-48:240:2.5cm);
%         %% The outer nodes
%         \foreach \x in {36,72,...,360}
%             \shade[ball color=black](\x:5) circle(4pt);
%         \foreach \nodes in {12,24,...,360}
%             \shade[ball color=black](\nodes:3.5) circle(4pt);
%         %%% The connecting nodes
%         \foreach \angle in {-48,-12,...,240}
%             \draw(\angle:2.5) --++(\angle:0.9cm);
%         %%% outer interconnects
%         \foreach \angle in {-24,12,...,306}
%             \draw(\angle:3.6) --++(\angle:0.9cm);
%         \foreach \y in {-24,12,...,240}
%             \shade[ball color=black](\y:4.5cm) circle(4pt);
    
%         %% outer most connections
%         \foreach \angle in{-36,0,...,306}
%             \draw(\angle:4.9cm) --(\angle:4.7cm) [rotate=\angle]arc(0:180:0.20cm);
%         \foreach \angle in{-36,0,...,306}
%             \draw(\angle:4.3cm) --(\angle:3.6cm);
%         %% Outer connects and leads
%         \shade[ball color=black](276:6) circle(4pt);
%         \draw(276:6)circle(4pt)--(276:5.2)[rotate=276]arc(0:180:0.25cm);
%         \draw(276:7)node {$\mathbf{K_0}$};
%         \draw(276:4.2)[rotate=276]arc(180:360:0.25cm);
%         \draw(276:4.2)--(276:3.5);
    
%         %% Exploitation of circular symmetry of the required figure
    
%         {[rotate=72]
%             \shade[ball color=black](276:6) circle(4pt);
%             \draw(276:6)circle(4pt)--(276:5.2)[rotate=276]arc(0:180:0.25cm);
%             \draw(270:6)node {$\mathbf{K_1-K_9}$};
%             \draw(276:4.2)[rotate=276]arc(180:360:0.25cm);%%%
%             \draw(276:4.2)--(276:3.5);
%         }
    
%         {[rotate=-48]
%             \shade[ball color=black](276:6) circle(4pt);
%             \draw(276:6)circle(4pt)--(276:5.2)[rotate=276]arc(0:180:0.20cm);
%             \draw(276:7)node {$\mathbf{g_2}$};
%             \draw(276:4.8)--(276:4.5);
%         }
    
%         \draw(180:5)--(180:6);
%         \shade[ball color=black](180:6) circle(4pt);
%         \draw(180:6.5)node{$\mathbf{g_1}$};
%     \end{tikzpicture}

%  %*----------- notes
%     \note[item]{Notes can help you to remember important information. Turn on the notes option.}
% \end{frame}
% %-
